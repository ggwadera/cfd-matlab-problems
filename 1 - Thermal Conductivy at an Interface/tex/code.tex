\section{Código}

\begin{minted}[
    fontsize=\small,
    linenos,
    fontfamily=courier
]{matlab}
%% Parâmetros e Constantes

k_A = 10;       k_B = 1;        % Condutividade - W/(m*K)
L_A = 0.069;    L_B = 0.031;    % Comprimento - m
N_A = 3;        N_B = N_A;      % Número de volumes em A e B
h   = 100;                      % Coeficiente convectivo - W/(m^2*K)
T_inf = 40;                     % Temperatura do fluido - °C
q_presc = 6000;                 % Fluxo prescrito - W/m^2
Area = 1;                       % Área da face - m^2
exato = [327.4, 100];           % Solução exata

%% Discretização e Inicialização

dx_A = L_A / (N_A - 0.5);       % Distância entre pontos em A
dx_B = L_B / (N_B - 0.5);       % Distância entre pontos em B
N    = N_A + N_B;               % Número total de volumes
dx   = [ones(1, N_A) .* dx_A, ones(1, N_B) .* dx_B];
dx_w = [0, ones(1, N_A-1) .* dx_A./2, ...
        ones(1, N_B).*dx_B./2];      % distâncias entre cada P e w
dx_e = [ones(1, N_A) .* dx_A./2, ...
        ones(1, N_B-1).*dx_B./2, 0]; % distâncias entre cada P e e
k = [ones(1, N_A) .* k_A, ones(1, N_B) .* k_B];  % k em cada ponto
A = zeros(N, N);                % Matriz dos coeficientes
b = zeros(N, 1);                % Vetor dos termos independentes

%% Resolução

% Construção do sistema de equações
for i = 1:N
    % Condutividade nas interfaces (kw e ke)
    if i == 1
        fe = dx_w(i+1) / (dx_e(i) + dx_w(i+1));
        ke = 1 / ((1 - fe) / k(i) + fe / k(i+1));   % resistências
        %ke = fe * k(i) + (1 - fe) * k(i+1);        % linear
    elseif i == N
        fw = dx_e(i-1) / (dx_w(i) + dx_e(i-1));
        kw = 1 / ((1 - fw) / k(i) + fw / k(i-1));   % resistências
        %kw = fw * k(i) + (1 - fw) * k(i-1);        % linear
    else
        fe = dx_w(i+1) / (dx_e(i) + dx_w(i+1));
        fw = dx_e(i-1) / (dx_w(i) + dx_e(i-1));
        ke = 1 / ((1 - fe) / k(i) + fe / k(i+1));   % resistências
        %ke = fe * k(i) + (1 - fe) * k(i+1);        % linear
        kw = 1 / ((1 - fw) / k(i) + fw / k(i-1));   % resistências
        %kw = fw * k(i) + (1 - fw) * k(i-1);        % linear
    end
 
 
 
    if i == 1
        % Fronteira com fluxo prescrito
        dx_PE = dx_e(i) + dx_w(i+1);    % distância entre P e E
        ae = ke * Area / dx_PE;
        ap = ae;
        b(i) = q_presc * Area;
        A(i, [i, i+1]) = [ap, -ae];
    elseif i == N
        % Fronteira com fluxo convectivo
        dx_PW = dx_w(i) + dx_e(i-1);    % distância entre P e W
        aw    = kw * Area / dx_PW;
        ap    = aw + h * Area;
        b(i)  = h * Area * T_inf;
        A(i, [i-1, i]) = [-aw, ap];
    else
        % Pontos no interior da malha
        dx_PE = dx_e(i) + dx_w(i+1);
        dx_PW = dx_w(i) + dx_e(i-1);
        ae    = ke * Area / dx_PE;
        aw    = kw * Area / dx_PW;
        ap    = ae + aw;
        A(i, [i-1, i, i+1]) = [-aw, ap, -ae];
    end
end

% Resolução do sistema de equações
T = A \ b;
\end{minted}